\documentclass[paper=a4, fontsize=10pt]{resume}
\usepackage{hyperref}

\usepackage[utf8]{inputenc}
\usepackage[T1]{fontenc}
\usepackage[default]{lato}

\usepackage{xcolor}
\usepackage[none]{hyphenat}

% If more space is needed margin can be removed
% \geometry{top=0cm, bottom=0cm}

\begin{document}

\name{Kaiqi Liang}
\tagline{}
\personalinfo{
	\begin{tabular}{ll}
		\phone{(61) 424180285} & \email{kaiqi.liang@outlook.com} \\
		\location{Perth, Western Australia} & \homepage{http://kaiqi-liang.web.app/} \\
		\linkedin{linkedin.com/in/kaiqiliang} & \github{github.com/Kaiqi-Liang}
	\end{tabular}
}

\AtBeginEnvironment{itemize}{\small}

%% Provide the file name containing the sidebar contents as an optional parameter to \cvsection.
%% You can always just use \marginpar{...} if you do
%% not need to align the top of the contents to any
%% \cvsection title in the "main" bar.

\sidebar{
	\makecvheader

	\cvsection{EDUCATION}
		\cvsideevent{Master of Information Technology}
			{University of Western Australia (UWA)}
			{February 2022 -- December 2023}
		\wam{79.250}
		\begin{itemize}
			\item 2022: Computer Science Student Club (CSSC) Posgraduate Representative
		\end{itemize}

		\divider

		\cvsideevent{Bachelor of Computer Science}
			{University of New South Wales (UNSW)}
			{February 2018 -- December 2021}
		\wam{78.042}
		\begin{itemize}
			\item 2021: Computer Science and Engineering (CSE) Society Executive, CSE Student Representative
			\item 2020: CSE Revue Social Director
			\item 2019: CSE Society Peer Mentor, CSESoc Media Subcommittee
		\end{itemize}

	\cvsection{Achievements}
		\begin{itemize}
			\notablesubject{100}{(=1st)}{COMP3121}{Algorithms \& Programming Techiques}
			\notablesubject{99}{(=1st)}{COMP1521}{Computer Systems Fundamentals}
			\notablesubject{98}{(2nd)}{COMP2511}{Object-Oriented Design \& Programming}
			\notablesubject{94}{(2nd)}{COMP2121}{Microprocessors \& Interfacing}
			\notablesubject{94}{}{COMP1531}{Software Engineering Fundamentals}
			\notablesubject{90}{}{COMP2521}{Data Structures and Algorithsm}
			\notablesubject{90}{}{COMP6080}{Web Front-End Programming}
			\notablesubject{87}{(=4th)}{CITS5501}{Software Testing \& Quality Assurance}
		\end{itemize}

	\cvsection{SKILLS}
		\cvskilltitle{Web Development}
		\begin{tabular}{llll}
			HTML & CSS & JavaScript & TypeScript \\
			React & MoBX & Vue
		\end{tabular}

		\medskip\normalsize

		\cvskilltitle{Programming Languages}
		\begin{tabular}{llll}
			C & C++ & C\# & Java \\
			Python & R & Rust & Perl \\
			VHDL & Assembly & Shell Script & SQL \\
			MATLAB & Dafny
		\end{tabular}

		\medskip\normalsize

		\cvskilltitle{Other Technologies}
		\begin{tabular}{llll}
			AWS & Git & Unity & Unreal Engine \\
			OpenGL & TensorFlow & scikit-learn
		\end{tabular}
}

\mainbar{
	\cvsection{EXPERIENCE}
		\cvsideevent{Academic Tutor}
			{UNSW School of Computer Science \& Engineering}
			{February 2019 -- Present}
		\begin{itemize}
			\item Explain technical concepts and go through coding questions in tutorials
			\item Guide students through the problem-solving and debugging process in computer laboratories and during help sessions
			\item Develop assignments and tutorials exercises
		\end{itemize}

		\divider

		\cvsideevent{Software Engineer Intern (Backend)}
			{Google}
			{November 2022 -- February 2023}
		\begin{itemize}
			\item Created a backend API in \technology{C++} to support emojis on Chrome OS
		\end{itemize}

		\divider

		\cvsideevent{International Student Engagement Ambassador}
			{UWA Office of Deputy Vice-Chancellor (Education)}
			{August 2022 -- November 2022}
		\begin{itemize}
			\item Wrote \technology{Python} script that invokes a \technology{node.js} script to streamline the process of cross checking student information and their bank details using the \technology{pandas} library and the \technology{pdf-parse} node module
			\item Organised excursions and social events tailored for international students
		\end{itemize}

		\divider

		\cvsideevent{NuraGili Indigenous Tuition Program Tutor}
			{UNSW Faculty of Deputy Vice-Chancellor (Academic)}
			{September 2022 -- November 2022}
		\begin{itemize}
			\item Provided one-on-one academic assistance for Digital Circuits and Systems (COMP3222) which uses \technology{VHDL} and Solving Modern Programming Problems with \technology{Rust} (COMP6991)
		\end{itemize}

		\divider

		% \cvsideevent{Professional Computing (CITS3200) Project Auditor}
		% 	{UWA School of Physics Maths \& Computing}
		% 	{August 2022 -- November 2022}
		% \begin{itemize}
		% 	\item Attended team meetings and commented on team processes
		% 	\item Checked the veracity of claimed hours against evidence on \technology{GitHub}
		% \end{itemize}

		% \divider

		\cvsideevent{Software Engineer Intern (Frontend)}
			{Canva}
			{December 2021 -- February 2022}
		\begin{itemize}
			\item Enhanced existing feature on Canva with added functionality to improve user experience
			\item Followed Canva frontend design patterns to produce good quality and consistent code in \technology{React TypeScript} with \technology{MobX}
		\end{itemize}

	\cvsection{PROJECTS}
		\cvevent{Tax Calculator}{}{}{}
			{https://kaiqi-liang.github.io/Tax-Calculator/}{}
		\begin{itemize}
			\item A calculator that determines the tax category and the amount of tax withheld based on your income salary
			\item Built frontend in \technology{React TypeScript} and deployed on \technology{GitHub.io}, backend in \technology{TypeScript} and deployed on \technology{Google App Engine}
		\end{itemize}

		\divider

		\cvevent{Chrome Extension}{}{}{}
			{http://bit.ly/YoutubeDistractionDisabler}{}
		\begin{itemize}
			\item A Chrome Extension for YouTube that removes distracting features
			\item Built in \technology{TypeScript} and compiled to \technology{Vanilla JavaScript} before publishing to \technology{Chrome Web Store}
		\end{itemize}
}

\makebody

\end{document}
